\documentclass[a4paper,10pt]{article}
 
\usepackage[english]{babel}
\usepackage[T1]{fontenc}
\usepackage[ansinew]{inputenc}
 
\usepackage{lmodern}	% font definition
\usepackage{amsmath}	% math fonts
\usepackage{amsthm}
\usepackage{amsfonts}
 
\usepackage{tikz}
 
%%%<
\usepackage{verbatim}
\usepackage[active,tightpage]{preview}
\PreviewEnvironment{tikzpicture}
\setlength\PreviewBorder{5pt}%
%%%>
 
\begin{comment}
:Title: Kalman Filter System Model
:Slug: kalman-filter
:Author: Burkart Lingner
 
This is the system model of the (linear) Kalman filter.
 
\end{comment}
 
 
\usetikzlibrary{decorations.pathmorphing} % noisy shapes
\usetikzlibrary{fit}					% fitting shapes to coordinates
\usetikzlibrary{backgrounds}	% drawing the background after the foreground
 
\begin{document}
 
\begin{figure}[htbp]
\centering
% The state vector is represented by a blue circle.
% "minimum size" makes sure all circles have the same size
% independently of their contents.
\tikzstyle{state}=[circle,label=below:$r$,
                                    thick,
                                    minimum size=1.2cm,
                                    draw=blue!80,
                                    fill=blue!20]
\tikzstyle{state2}=[circle,label=below:$e$,
                                    thick,
                                    minimum size=1.2cm,
                                    draw=blue!80,
                                    fill=blue!20]
\tikzstyle{state3}=[circle,label=below:$a$,
                                    thick,
                                    minimum size=1.2cm,
                                    draw=blue!80,
                                    fill=blue!20]
\tikzstyle{state4}=[circle,label=below:$d$,
                                    thick,
                                    minimum size=1.2cm,
                                    draw=blue!80,
                                    fill=blue!20]
% The measurement vector is represented by an orange circle.
\tikzstyle{measurement}=[circle,
                                                thick,
                                                minimum size=1.2cm,
                                                draw=orange!80,
                                                fill=orange!25]
 
% The control input vector is represented by a purple circle.
\tikzstyle{input}=[circle,
                                    thick,
                                    minimum size=1.2cm,
                                    draw=purple!80,
                                    fill=purple!20]
 
% The input, state transition, and measurement matrices
% are represented by gray squares.
% They have a smaller minimal size for aesthetic reasons.
\tikzstyle{matrx}=[rectangle,
                                    thick,
                                    minimum size=1cm,
                                    draw=gray!80,
                                    fill=gray!20]
 
% The system and measurement noise are represented by yellow
% circles with a "noisy" uneven circumference.
% This requires the TikZ library "decorations.pathmorphing".
\tikzstyle{noise}=[circle,
                                    thick,
                                    minimum size=1.2cm,
                                    draw=yellow!85!black,
                                    fill=yellow!40,
                                    decorate,
                                    decoration={random steps,
                                                            segment length=2pt,
                                                            amplitude=2pt}]
 
% Everything is drawn on underlying gray rectangles with
% rounded corners.
\tikzstyle{background}=[rectangle,
                                                fill=gray!10,
                                                inner sep=0.2cm,
                                                rounded corners=5mm]
 
\begin{tikzpicture}[>=latex,text height=1.5ex,text depth=0.25ex]
    % "text height" and "text depth" are required to vertically
    % align the labels with and without indices.
 
  % The various elements are conveniently placed using a matrix:
  \matrix[row sep=0.5cm,column sep=0.5cm] {
    % First line: Control input
    &
    \node (y_0) [measurement] {$\mathbf{y}_{0}$}; &
\node (y_1) [measurement] {$\mathbf{y}_{1}$}; &
\node (y_2) [measurement] {$\mathbf{y}_{2}$}; &
        \node (y_3) [measurement] {$\mathbf{y}_{3}$}; &
        \\
        % Second line: System noise & input matrix
        \node ({S_{3}}') [input] {$\mathbf{S_{3}}'$}; &
        \node ({A_{0}}') [matrx] {$\mathbf{A_{0}}'$}; &
        \node ({A_{1}}') [matrx] {$\mathbf{A_{1}}'$}; &
        \node ({A_{2}}') [matrx] {$\mathbf{A_{2}}'$}; &
        \node ({A_{3}}') [matrx] {$\mathbf{A_{3}}'$}; &
     \node ({S_{0}}') [input] {$\mathbf{S_{0}}'$}; &
        \\
        \node ({S_{0}}) [input] {$\mathbf{S_{0}}$}; &
        \node ({A_{0}}) [matrx] {$\mathbf{A_{0}}$}; &
        \node ({A_{1}}) [matrx] {$\mathbf{A_{1}}$}; &
        \node ({A_{2}}) [matrx] {$\mathbf{A_{2}}$}; &
        \node ({A_{3}}) [matrx] {$\mathbf{A_{3}}$}; &
     \node ({S_{3}}) [input] {$\mathbf{S_{3}}$}; &
        \\
        % Fifth line: 输入
         &
        \node (x_0) [state]{$\mathbf{x}_{0}$}; &
 
         \node (x_1) [state2]{$\mathbf{x}_{1}$}; &
 
        \node (x_2)   [state3]{$\mathbf{x}_{2}$}; &
 
        \node (x_3) [state4]{$\mathbf{x}_{3}$}; &
 
    \\
    };
 
    % The diagram elements are now connected through arrows:
    \path[->]
        ({S_{0}}') edge[thick] ({A_{3}}')
        ({A_{3}}') edge[thick] ({A_{2}}')
        ({A_{2}}') edge[thick] ({A_{1}}')
        ({A_{1}}') edge[thick] ({A_{0}}')
        ({A_{0}}') edge[thick] ({S_{3}}')
 
        ({S_{0}}) edge[thick] ({A_{0}})
        ({A_{0}}) edge[thick] ({A_{1}})
        ({A_{1}}) edge[thick] ({A_{2}})
        ({A_{2}}) edge[thick] ({A_{3}})
        ({A_{3}}) edge[thick] ({S_{3}})
 
        (x_0) edge ({A_{0}})
        (x_1) edge ({A_{1}})
        (x_2) edge ({A_{2}})
        (x_3) edge ({A_{3}})
        ({A_{0}}') edge (y_0)
        ({A_{1}}') edge (y_1)
        ({A_{2}}') edge (y_2)
        ({A_{3}}') edge (y_3)
 
        (x_0) edge[->,bend right=37,green]	({A_{0}}')
        (x_1) edge[->,bend right=37,green]	({A_{1}}')
        (x_2) edge[->,bend right=37,green]	({A_{2}}')
        (x_3) edge[->,bend right=37,green]	({A_{3}}')
        ({A_{3}}) edge[->,bend right=37,green]	(y_3)
        ({A_{2}}) edge[->,bend right=37,green]	(y_2)
        ({A_{1}}) edge[->,bend right=37,green]	(y_1)
        ({A_{0}}) edge[->,bend right=37,green]	(y_0)
        ;
 
    % Now that the diagram has been drawn, background rectangles
    % can be fitted to its elements. This requires the TikZ
    % libraries "fit" and "background".
    % Control input and measurement are labeled. These labels have
    % not been translated to English as "Measurement" instead of
    % "Messung" would not look good due to it being too long a word.
 
 
\end{tikzpicture}
 
\caption{Kalman filter system model}
\end{figure}
 
This is the system model of the (linear) Kalman filter. At each time
step the state vector $\mathbf{x}_k$ is propagated to the new state
estimation $\mathbf{x}_{k+1}$ by multiplication with the constant state
transition matrix $\mathbf{A}$. The state vector $\mathbf{x}_{k+1}$ is
additionally influenced by the control input vector $\mathbf{u}_{k+1}$
multiplied by the input matrix $\mathbf{B}$, and the system noise vector
$\mathbf{w}_{k+1}$. The system state cannot be measured directly. The
measurement vector $\mathbf{z}_k$ consists of the information contained
within the state vector $\mathbf{x}_k$ multiplied by the measurement
matrix $\mathbf{H}$, and the additional measurement noise $\mathbf{v}_k$.
 
\end{document}