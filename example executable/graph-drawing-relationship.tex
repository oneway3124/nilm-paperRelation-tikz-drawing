% Graph drawing algorithms do the tough work of computing a layout of a graph for you. TikZ comes with
% powerful such algorithms, but you can also implement new algorithms in the Lua programming language.
% From <pdfmanual.pdf> part iv Graph Drawing chapter 27 
% wang.david
\documentclass[tikz,border=10pt]{standalone}


\begin{document}

\usetikzlibrary {arrows.meta,graphs,graphdrawing} \usegdlibrary {layered}
\tikz [nodes={text height=.7em, text depth=.2em,
draw=black!20, thick, fill=white, font=\footnotesize}, >={Stealth[round,sep]}, rounded corners, semithick]
\graph [layered layout, level distance=1cm, sibling sep=.5em, sibling distance=1cm] {
"5th Edition" -> { "6th Edition", "PWB 1.0" };
"6th Edition" -> { "LSX" [>child anchor=45], "1 BSD", "Mini Unix", "Wollongong", "Interdata" };
"Interdata" -> { "Unix/TS 3.0", "PWB 2.0", "7th Edition" };
"7th Edition" -> { "8th Edition", "32V", "V7M", "Ultrix-11", "Xenix", "UniPlus+" };
"V7M" -> "Ultrix-11";
"8th Edition" -> "9th Edition";
"1 BSD" -> "2 BSD" -> "2.8 BSD" -> { "Ultrix-11", "2.9 BSD" };
"32V" -> "3 BSD" -> "4 BSD" -> "4.1 BSD" -> { "4.2 BSD", "2.8 BSD", "8th Edition" };
"4.2 BSD" -> { "4.3 BSD", "Ultrix-32" };
"PWB 1.0" -> { "PWB 1.2" -> "PWB 2.0", "USG 1.0" -> { "CB Unix 1", "USG 2.0" }};
"CB Unix 1" -> "CB Unix 2" -> "CB Unix 3" -> { "Unix/TS++", "PDP-11 Sys V" };
{ "USG 2.0" -> "USG 3.0", "PWB 2.0", "Unix/TS 1.0" } -> "Unix/TS 3.0"; { "Unix/TS++", "CB Unix 3", "Unix/TS 3.0" } -> "TS 4.0" -> "System V.0" -> "System V.2" -> "System V.3";
};


\end{document}
