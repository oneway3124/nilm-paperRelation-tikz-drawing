\documentclass{article}

\usepackage{tikz}
\begin{document}
\pagestyle{empty}

\def\layersep{2.5cm}

\begin{tikzpicture}[shorten >=1pt,->,draw=black!50, node distance=\layersep]
    \tikzstyle{every pin edge}=[<-,shorten <=1pt]
    \tikzstyle{neuron}=[circle,fill=black!25,minimum size=17pt,inner sep=0pt]
    \tikzstyle{input neuron}=[neuron, fill=green!50];
    \tikzstyle{output neuron}=[neuron, fill=red!50];

    \tikzstyle{hidden neuron1}=[neuron, fill=blue!50];
    \tikzstyle{hidden neuron2}=[neuron, fill=blue!50];
    \tikzstyle{hidden neuron3}=[neuron, fill=blue!50];

    \tikzstyle{annot} = [text width=4em, text centered]

    % Draw the input layer nodes
    \foreach \name / \y in {1,...,4}
    % This is the same as writing \foreach \name / \y in {1/1,2/2,3/3,4/4}
        \node[input neuron, pin=left:Input \#\y] (I-\name) at (0,-\y) {};

    % Draw the hidden layer1 nodes
    \foreach \name / \y in {1,...,5}
        \path[yshift=0.5cm]
            node[hidden neuron1] (H1-\name) at (\layersep,-\y cm) {};
   %Draw the hidden layer2 nodes
    \foreach \name / \y in {1,...,5}
        \path[yshift=0.5cm]
            node[hidden neuron2,right of=H1] (H2-\name) at (\layersep,-\y cm){};
    %Draw the hidden layer3 nodes
    \foreach \name / \y in {1,...,5}
        \path[yshift=0.5cm]
            node[hidden neuron3,right of=H2] (H3-\name) at (2*\layersep,-\y cm){};

    % Draw the output layer node
    \node[output neuron,pin={[pin edge={->}]right:Output}, right of=H3-3] (O) {};

    % Connect every node in the input layer with every node in the
    % hidden layer.
    \foreach \source in {1,...,4}
        \foreach \dest in {1,...,5}
            \path (I-\source) edge (H1-\dest);
    %
    \foreach \source in {1,...,5}
        \foreach \dest in {1,...,5}
            \path (H1-\source) edge (H2-\dest);
    %
    \foreach \source in {1,...,5}
        \foreach \dest in {1,...,5}
            \path (H2-\source) edge (H3-\dest);
    % Connect every node in the hidden layer with the output layer
    \foreach \source in {1,...,5}
        \path (H3-\source) edge (O);

    % Annotate the layers
    \node[annot,above of=H1-1, node distance=1cm] (hl) {Hidden layer1};
    \node[annot,left of=hl] {Input layer};
    \node[annot,right of=hl] (h2){Hidden layer2};
    \node[annot,right of=h2] (h3){Hidden layer3};
    \node[annot,right of=h3] {Output layer};
\end{tikzpicture}
% End of code
\end{document}